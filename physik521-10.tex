% Copyright © 2012-2014 Martin Ueding <dev@martin-ueding.de>

% This is my general purpose LaTeX header file for writing German documents.
% Ideally, you include this using a simple ``% Copyright © 2012-2014 Martin Ueding <dev@martin-ueding.de>

% This is my general purpose LaTeX header file for writing German documents.
% Ideally, you include this using a simple ``% Copyright © 2012-2014 Martin Ueding <dev@martin-ueding.de>

% This is my general purpose LaTeX header file for writing German documents.
% Ideally, you include this using a simple ``\input{header.tex}`` in your main
% document and start with ``\title`` and ``\begin{document}`` afterwards.

% If you need to add additional packages, I recommend not doing this in this
% file, but in your main document. That way, you can just drop in a new
% ``header.tex`` and get all the new commands without having to merge manually.

% Since this file encorporates a CC-BY-SA fragment, this whole files is
% licensed under the CC-BY-SA license.

\documentclass[11pt, ngerman, fleqn, DIV=15]{scrartcl}

\usepackage{graphicx}

\setkomafont{caption}{\sffamily}
\setkomafont{captionlabel}{\usekomafont{caption}}

%%%%%%%%%%%%%%%%%%%%%%%%%%%%%%%%%%%%%%%%%%%%%%%%%%%%%%%%%%%%%%%%%%%%%%%%%%%%%%%
%                                Locale, date                                 %
%%%%%%%%%%%%%%%%%%%%%%%%%%%%%%%%%%%%%%%%%%%%%%%%%%%%%%%%%%%%%%%%%%%%%%%%%%%%%%%

\usepackage{babel}
\usepackage[iso]{isodate}

%%%%%%%%%%%%%%%%%%%%%%%%%%%%%%%%%%%%%%%%%%%%%%%%%%%%%%%%%%%%%%%%%%%%%%%%%%%%%%%
%                          Margins and other spacing                          %
%%%%%%%%%%%%%%%%%%%%%%%%%%%%%%%%%%%%%%%%%%%%%%%%%%%%%%%%%%%%%%%%%%%%%%%%%%%%%%%

\usepackage[parfill]{parskip}
\usepackage{setspace}
\usepackage[activate]{microtype}

\setlength{\columnsep}{2cm}

%%%%%%%%%%%%%%%%%%%%%%%%%%%%%%%%%%%%%%%%%%%%%%%%%%%%%%%%%%%%%%%%%%%%%%%%%%%%%%%
%                                    Color                                    %
%%%%%%%%%%%%%%%%%%%%%%%%%%%%%%%%%%%%%%%%%%%%%%%%%%%%%%%%%%%%%%%%%%%%%%%%%%%%%%%

\usepackage[usenames, dvipsnames]{xcolor}

\colorlet{darkred}{red!70!black}
\colorlet{darkblue}{blue!70!black}
\colorlet{darkgreen}{green!40!black}

%%%%%%%%%%%%%%%%%%%%%%%%%%%%%%%%%%%%%%%%%%%%%%%%%%%%%%%%%%%%%%%%%%%%%%%%%%%%%%%
%                         Font and font like settings                         %
%%%%%%%%%%%%%%%%%%%%%%%%%%%%%%%%%%%%%%%%%%%%%%%%%%%%%%%%%%%%%%%%%%%%%%%%%%%%%%%

% This replaces all fonts with Bitstream Charter, Bitstream Vera Sans and
% Bitstream Vera Mono. Math will be rendered in Charter.
\usepackage[charter, greekuppercase=italicized]{mathdesign}
\usepackage{beramono}
\usepackage{berasans}

% Bold, sans-serif tensors. This fragment is taken from “egreg” from
% http://tex.stackexchange.com/a/82747/8945 and licensed under `CC-BY-SA
% <https://creativecommons.org/licenses/by-sa/3.0/>`_.
\usepackage{bm}
\DeclareMathAlphabet{\mathsfit}{\encodingdefault}{\sfdefault}{m}{sl}
\SetMathAlphabet{\mathsfit}{bold}{\encodingdefault}{\sfdefault}{bx}{sl}
\newcommand{\tens}[1]{\bm{\mathsfit{#1}}}

% Bold vectors.
\renewcommand{\vec}[1]{\boldsymbol{#1}}

%%%%%%%%%%%%%%%%%%%%%%%%%%%%%%%%%%%%%%%%%%%%%%%%%%%%%%%%%%%%%%%%%%%%%%%%%%%%%%%
%                               Input encoding                                %
%%%%%%%%%%%%%%%%%%%%%%%%%%%%%%%%%%%%%%%%%%%%%%%%%%%%%%%%%%%%%%%%%%%%%%%%%%%%%%%

\usepackage[T1]{fontenc}
\usepackage[utf8]{inputenc}

%%%%%%%%%%%%%%%%%%%%%%%%%%%%%%%%%%%%%%%%%%%%%%%%%%%%%%%%%%%%%%%%%%%%%%%%%%%%%%%
%                         Hyperrefs and PDF metadata                          %
%%%%%%%%%%%%%%%%%%%%%%%%%%%%%%%%%%%%%%%%%%%%%%%%%%%%%%%%%%%%%%%%%%%%%%%%%%%%%%%

\usepackage{hyperref}

% This sets the author in the properties of the PDF as well. If you want to
% change it, just override it with another ``\hypersetup`` call.
\hypersetup{
	breaklinks=false,
	citecolor=darkgreen,
	colorlinks=true,
	linkcolor=darkblue,
	menucolor=black,
	pdfauthor={Martin Ueding},
	urlcolor=darkblue,
}

%%%%%%%%%%%%%%%%%%%%%%%%%%%%%%%%%%%%%%%%%%%%%%%%%%%%%%%%%%%%%%%%%%%%%%%%%%%%%%%
%                               Math Operators                                %
%%%%%%%%%%%%%%%%%%%%%%%%%%%%%%%%%%%%%%%%%%%%%%%%%%%%%%%%%%%%%%%%%%%%%%%%%%%%%%%

% AMS environments like ``align`` and theorems like ``proof``.
\usepackage{amsmath}
\usepackage{amsthm}

% Common math constructs like partial derivatives.
\usepackage{commath}

% Physical units.
\usepackage[output-decimal-marker={,}]{siunitx}

% Since I use mathdesign with italic uppercase greek characters, the Ohm unit will be displayed with an italic Ω by default. Units have to be roman, so this forces it the right way.
\DeclareSIUnit{\ohm}{$\Omegaup$}
\DeclareSIUnit{\division}{DIV}
\DeclareSIUnit{\voltss}{$\mathrm{V_{SS}}$}

% Word like operators.
\DeclareMathOperator{\acosh}{arcosh}
\DeclareMathOperator{\arcosh}{arcosh}
\DeclareMathOperator{\arcsinh}{arsinh}
\DeclareMathOperator{\arsinh}{arsinh}
\DeclareMathOperator{\asinh}{arsinh}
\DeclareMathOperator{\card}{card}
\DeclareMathOperator{\csch}{csch}
\DeclareMathOperator{\diam}{diam}
\DeclareMathOperator{\sech}{sech}
\renewcommand{\Im}{\mathop{{}\mathrm{Im}}\nolimits}
\renewcommand{\Re}{\mathop{{}\mathrm{Re}}\nolimits}

% Fourier transform.
\DeclareMathOperator{\fourier}{\ensuremath{\mathcal{F}}}

% Roman versions of “e” and “i” to serve as Euler's number and the imaginary
% constant.
\newcommand{\eup}{\mathrm e}
\newcommand{\iup}{\mathrm i}

% Symbols for the various mathematical fields (natural numbers, integers,
% rational numbers, real numbers, complex numbers).
\newcommand{\C}{\ensuremath{\mathbb C}}
\newcommand{\N}{\ensuremath{\mathbb N}}
\newcommand{\Q}{\ensuremath{\mathbb Q}}
\newcommand{\R}{\ensuremath{\mathbb R}}
\newcommand{\Z}{\ensuremath{\mathbb Z}}

% Shape like operators.
\DeclareMathOperator{\dalambert}{\Box}
\DeclareMathOperator{\laplace}{\bigtriangleup}
\newcommand{\curl}{\vnabla \times}
\newcommand{\divergence}[1]{\inner{\vnabla}{#1}}
\newcommand{\Divergence}[1]{\Inner{\vnabla}{#1}}
\newcommand{\vnabla}{\vec \nabla}

\newcommand{\half}{\frac 12}

% Unit vector (German „Einheitsvektor“).
\newcommand{\ev}{\hat{\vec e}}

% Mathematician's notation for the inner (scalar, dot) product.
\newcommand{\bracket}[1]{\langle #1 \rangle}
\newcommand{\Bracket}[1]{\left\langle #1 \right\rangle}
\newcommand{\inner}[2]{\bracket{#1, #2}}
\newcommand{\Inner}[2]{\Bracket{#1, #2}}

% Placeholders.
\newcommand{\fehlt}{\textcolor{darkred}{Hier fehlen noch Inhalte.}}
\newcommand{\messwert}{\textcolor{blue}{\square}}
\newcommand{\punkte}{\phantom{xxxxx}}

% Separator for equations on a single line.
\newcommand{\eqnsep}{,\quad}

% Quantum Mechanics.
\usepackage{braket}

% Thermodynamic partial derivative.
\newcommand\tdpd[3]{\del{\dpd{#1}{#2}}_{#3}}

%%%%%%%%%%%%%%%%%%%%%%%%%%%%%%%%%%%%%%%%%%%%%%%%%%%%%%%%%%%%%%%%%%%%%%%%%%%%%%%
%                                  Headings                                   %
%%%%%%%%%%%%%%%%%%%%%%%%%%%%%%%%%%%%%%%%%%%%%%%%%%%%%%%%%%%%%%%%%%%%%%%%%%%%%%%

% This will set fancy headings to the top of the page. The page number will be
% accompanied by the total number of pages. That way, you will know if any page
% is missing.
%
% If you do not want this for your document, you can just use
% ``\pagestyle{plain}``.

\usepackage{scrpage2}

\pagestyle{scrheadings}
\automark{section}
\chead{}
\ihead{}
\ohead{\rightmark}
\setheadsepline{.4pt}

%%%%%%%%%%%%%%%%%%%%%%%%%%%%%%%%%%%%%%%%%%%%%%%%%%%%%%%%%%%%%%%%%%%%%%%%%%%%%%%
%                            Bibliography (BibTeX)                            %
%%%%%%%%%%%%%%%%%%%%%%%%%%%%%%%%%%%%%%%%%%%%%%%%%%%%%%%%%%%%%%%%%%%%%%%%%%%%%%%

\newcommand{\bibliographyfile}{../../zentrale_BibTeX/Central}

\usepackage[
    backend=bibtex,
    style=authoryear-icomp,
    %isbn=false,
    %pagetracker=false,
    %maxbibnames=50,
    %maxcitenames=2,
    %autocite=inline,
    %block=space,
    %backref=false,
    %backrefstyle=three+,
    %date=short,
    hyperref=true
]{biblatex}

\setlength{\bibitemsep}{.7em}
\setlength{\bibhang}{4ex}

\IfFileExists{\bibliographyfile}{
    \bibliography{\bibliographyfile}
}{}

%%%%%%%%%%%%%%%%%%%%%%%%%%%%%%%%%%%%%%%%%%%%%%%%%%%%%%%%%%%%%%%%%%%%%%%%%%%%%%%
%                                Abbreviations                                %
%%%%%%%%%%%%%%%%%%%%%%%%%%%%%%%%%%%%%%%%%%%%%%%%%%%%%%%%%%%%%%%%%%%%%%%%%%%%%%%

\newcommand{\dhabk}{\mbox{d.\,h.}}

%%%%%%%%%%%%%%%%%%%%%%%%%%%%%%%%%%%%%%%%%%%%%%%%%%%%%%%%%%%%%%%%%%%%%%%%%%%%%%%
%                                  Licences                                   %
%%%%%%%%%%%%%%%%%%%%%%%%%%%%%%%%%%%%%%%%%%%%%%%%%%%%%%%%%%%%%%%%%%%%%%%%%%%%%%%

\usepackage{ccicons}

\newcommand{\ccbysadetext}{%
	\begin{small}
		Dieses Werk bzw. Inhalt steht unter einer
		\href{http://creativecommons.org/licenses/by-sa/3.0/deed.de}{%
			Creative Commons Namensnennung - Weitergabe unter gleichen
		Bedingungen 3.0 Unported Lizenz}.
	\end{small}
}

\newcommand{\ccbysadetitle}{%
	Lizenz: \href{http://creativecommons.org/licenses/by-sa/3.0/deed.de}
	{CC-BY-SA 3.0 \ccbysa}
}
`` in your main
% document and start with ``\title`` and ``\begin{document}`` afterwards.

% If you need to add additional packages, I recommend not doing this in this
% file, but in your main document. That way, you can just drop in a new
% ``header.tex`` and get all the new commands without having to merge manually.

% Since this file encorporates a CC-BY-SA fragment, this whole files is
% licensed under the CC-BY-SA license.

\documentclass[11pt, ngerman, fleqn, DIV=15]{scrartcl}

\usepackage{graphicx}

\setkomafont{caption}{\sffamily}
\setkomafont{captionlabel}{\usekomafont{caption}}

%%%%%%%%%%%%%%%%%%%%%%%%%%%%%%%%%%%%%%%%%%%%%%%%%%%%%%%%%%%%%%%%%%%%%%%%%%%%%%%
%                                Locale, date                                 %
%%%%%%%%%%%%%%%%%%%%%%%%%%%%%%%%%%%%%%%%%%%%%%%%%%%%%%%%%%%%%%%%%%%%%%%%%%%%%%%

\usepackage{babel}
\usepackage[iso]{isodate}

%%%%%%%%%%%%%%%%%%%%%%%%%%%%%%%%%%%%%%%%%%%%%%%%%%%%%%%%%%%%%%%%%%%%%%%%%%%%%%%
%                          Margins and other spacing                          %
%%%%%%%%%%%%%%%%%%%%%%%%%%%%%%%%%%%%%%%%%%%%%%%%%%%%%%%%%%%%%%%%%%%%%%%%%%%%%%%

\usepackage[parfill]{parskip}
\usepackage{setspace}
\usepackage[activate]{microtype}

\setlength{\columnsep}{2cm}

%%%%%%%%%%%%%%%%%%%%%%%%%%%%%%%%%%%%%%%%%%%%%%%%%%%%%%%%%%%%%%%%%%%%%%%%%%%%%%%
%                                    Color                                    %
%%%%%%%%%%%%%%%%%%%%%%%%%%%%%%%%%%%%%%%%%%%%%%%%%%%%%%%%%%%%%%%%%%%%%%%%%%%%%%%

\usepackage[usenames, dvipsnames]{xcolor}

\colorlet{darkred}{red!70!black}
\colorlet{darkblue}{blue!70!black}
\colorlet{darkgreen}{green!40!black}

%%%%%%%%%%%%%%%%%%%%%%%%%%%%%%%%%%%%%%%%%%%%%%%%%%%%%%%%%%%%%%%%%%%%%%%%%%%%%%%
%                         Font and font like settings                         %
%%%%%%%%%%%%%%%%%%%%%%%%%%%%%%%%%%%%%%%%%%%%%%%%%%%%%%%%%%%%%%%%%%%%%%%%%%%%%%%

% This replaces all fonts with Bitstream Charter, Bitstream Vera Sans and
% Bitstream Vera Mono. Math will be rendered in Charter.
\usepackage[charter, greekuppercase=italicized]{mathdesign}
\usepackage{beramono}
\usepackage{berasans}

% Bold, sans-serif tensors. This fragment is taken from “egreg” from
% http://tex.stackexchange.com/a/82747/8945 and licensed under `CC-BY-SA
% <https://creativecommons.org/licenses/by-sa/3.0/>`_.
\usepackage{bm}
\DeclareMathAlphabet{\mathsfit}{\encodingdefault}{\sfdefault}{m}{sl}
\SetMathAlphabet{\mathsfit}{bold}{\encodingdefault}{\sfdefault}{bx}{sl}
\newcommand{\tens}[1]{\bm{\mathsfit{#1}}}

% Bold vectors.
\renewcommand{\vec}[1]{\boldsymbol{#1}}

%%%%%%%%%%%%%%%%%%%%%%%%%%%%%%%%%%%%%%%%%%%%%%%%%%%%%%%%%%%%%%%%%%%%%%%%%%%%%%%
%                               Input encoding                                %
%%%%%%%%%%%%%%%%%%%%%%%%%%%%%%%%%%%%%%%%%%%%%%%%%%%%%%%%%%%%%%%%%%%%%%%%%%%%%%%

\usepackage[T1]{fontenc}
\usepackage[utf8]{inputenc}

%%%%%%%%%%%%%%%%%%%%%%%%%%%%%%%%%%%%%%%%%%%%%%%%%%%%%%%%%%%%%%%%%%%%%%%%%%%%%%%
%                         Hyperrefs and PDF metadata                          %
%%%%%%%%%%%%%%%%%%%%%%%%%%%%%%%%%%%%%%%%%%%%%%%%%%%%%%%%%%%%%%%%%%%%%%%%%%%%%%%

\usepackage{hyperref}

% This sets the author in the properties of the PDF as well. If you want to
% change it, just override it with another ``\hypersetup`` call.
\hypersetup{
	breaklinks=false,
	citecolor=darkgreen,
	colorlinks=true,
	linkcolor=darkblue,
	menucolor=black,
	pdfauthor={Martin Ueding},
	urlcolor=darkblue,
}

%%%%%%%%%%%%%%%%%%%%%%%%%%%%%%%%%%%%%%%%%%%%%%%%%%%%%%%%%%%%%%%%%%%%%%%%%%%%%%%
%                               Math Operators                                %
%%%%%%%%%%%%%%%%%%%%%%%%%%%%%%%%%%%%%%%%%%%%%%%%%%%%%%%%%%%%%%%%%%%%%%%%%%%%%%%

% AMS environments like ``align`` and theorems like ``proof``.
\usepackage{amsmath}
\usepackage{amsthm}

% Common math constructs like partial derivatives.
\usepackage{commath}

% Physical units.
\usepackage[output-decimal-marker={,}]{siunitx}

% Since I use mathdesign with italic uppercase greek characters, the Ohm unit will be displayed with an italic Ω by default. Units have to be roman, so this forces it the right way.
\DeclareSIUnit{\ohm}{$\Omegaup$}
\DeclareSIUnit{\division}{DIV}
\DeclareSIUnit{\voltss}{$\mathrm{V_{SS}}$}

% Word like operators.
\DeclareMathOperator{\acosh}{arcosh}
\DeclareMathOperator{\arcosh}{arcosh}
\DeclareMathOperator{\arcsinh}{arsinh}
\DeclareMathOperator{\arsinh}{arsinh}
\DeclareMathOperator{\asinh}{arsinh}
\DeclareMathOperator{\card}{card}
\DeclareMathOperator{\csch}{csch}
\DeclareMathOperator{\diam}{diam}
\DeclareMathOperator{\sech}{sech}
\renewcommand{\Im}{\mathop{{}\mathrm{Im}}\nolimits}
\renewcommand{\Re}{\mathop{{}\mathrm{Re}}\nolimits}

% Fourier transform.
\DeclareMathOperator{\fourier}{\ensuremath{\mathcal{F}}}

% Roman versions of “e” and “i” to serve as Euler's number and the imaginary
% constant.
\newcommand{\eup}{\mathrm e}
\newcommand{\iup}{\mathrm i}

% Symbols for the various mathematical fields (natural numbers, integers,
% rational numbers, real numbers, complex numbers).
\newcommand{\C}{\ensuremath{\mathbb C}}
\newcommand{\N}{\ensuremath{\mathbb N}}
\newcommand{\Q}{\ensuremath{\mathbb Q}}
\newcommand{\R}{\ensuremath{\mathbb R}}
\newcommand{\Z}{\ensuremath{\mathbb Z}}

% Shape like operators.
\DeclareMathOperator{\dalambert}{\Box}
\DeclareMathOperator{\laplace}{\bigtriangleup}
\newcommand{\curl}{\vnabla \times}
\newcommand{\divergence}[1]{\inner{\vnabla}{#1}}
\newcommand{\Divergence}[1]{\Inner{\vnabla}{#1}}
\newcommand{\vnabla}{\vec \nabla}

\newcommand{\half}{\frac 12}

% Unit vector (German „Einheitsvektor“).
\newcommand{\ev}{\hat{\vec e}}

% Mathematician's notation for the inner (scalar, dot) product.
\newcommand{\bracket}[1]{\langle #1 \rangle}
\newcommand{\Bracket}[1]{\left\langle #1 \right\rangle}
\newcommand{\inner}[2]{\bracket{#1, #2}}
\newcommand{\Inner}[2]{\Bracket{#1, #2}}

% Placeholders.
\newcommand{\fehlt}{\textcolor{darkred}{Hier fehlen noch Inhalte.}}
\newcommand{\messwert}{\textcolor{blue}{\square}}
\newcommand{\punkte}{\phantom{xxxxx}}

% Separator for equations on a single line.
\newcommand{\eqnsep}{,\quad}

% Quantum Mechanics.
\usepackage{braket}

% Thermodynamic partial derivative.
\newcommand\tdpd[3]{\del{\dpd{#1}{#2}}_{#3}}

%%%%%%%%%%%%%%%%%%%%%%%%%%%%%%%%%%%%%%%%%%%%%%%%%%%%%%%%%%%%%%%%%%%%%%%%%%%%%%%
%                                  Headings                                   %
%%%%%%%%%%%%%%%%%%%%%%%%%%%%%%%%%%%%%%%%%%%%%%%%%%%%%%%%%%%%%%%%%%%%%%%%%%%%%%%

% This will set fancy headings to the top of the page. The page number will be
% accompanied by the total number of pages. That way, you will know if any page
% is missing.
%
% If you do not want this for your document, you can just use
% ``\pagestyle{plain}``.

\usepackage{scrpage2}

\pagestyle{scrheadings}
\automark{section}
\chead{}
\ihead{}
\ohead{\rightmark}
\setheadsepline{.4pt}

%%%%%%%%%%%%%%%%%%%%%%%%%%%%%%%%%%%%%%%%%%%%%%%%%%%%%%%%%%%%%%%%%%%%%%%%%%%%%%%
%                            Bibliography (BibTeX)                            %
%%%%%%%%%%%%%%%%%%%%%%%%%%%%%%%%%%%%%%%%%%%%%%%%%%%%%%%%%%%%%%%%%%%%%%%%%%%%%%%

\newcommand{\bibliographyfile}{../../zentrale_BibTeX/Central}

\usepackage[
    backend=bibtex,
    style=authoryear-icomp,
    %isbn=false,
    %pagetracker=false,
    %maxbibnames=50,
    %maxcitenames=2,
    %autocite=inline,
    %block=space,
    %backref=false,
    %backrefstyle=three+,
    %date=short,
    hyperref=true
]{biblatex}

\setlength{\bibitemsep}{.7em}
\setlength{\bibhang}{4ex}

\IfFileExists{\bibliographyfile}{
    \bibliography{\bibliographyfile}
}{}

%%%%%%%%%%%%%%%%%%%%%%%%%%%%%%%%%%%%%%%%%%%%%%%%%%%%%%%%%%%%%%%%%%%%%%%%%%%%%%%
%                                Abbreviations                                %
%%%%%%%%%%%%%%%%%%%%%%%%%%%%%%%%%%%%%%%%%%%%%%%%%%%%%%%%%%%%%%%%%%%%%%%%%%%%%%%

\newcommand{\dhabk}{\mbox{d.\,h.}}

%%%%%%%%%%%%%%%%%%%%%%%%%%%%%%%%%%%%%%%%%%%%%%%%%%%%%%%%%%%%%%%%%%%%%%%%%%%%%%%
%                                  Licences                                   %
%%%%%%%%%%%%%%%%%%%%%%%%%%%%%%%%%%%%%%%%%%%%%%%%%%%%%%%%%%%%%%%%%%%%%%%%%%%%%%%

\usepackage{ccicons}

\newcommand{\ccbysadetext}{%
	\begin{small}
		Dieses Werk bzw. Inhalt steht unter einer
		\href{http://creativecommons.org/licenses/by-sa/3.0/deed.de}{%
			Creative Commons Namensnennung - Weitergabe unter gleichen
		Bedingungen 3.0 Unported Lizenz}.
	\end{small}
}

\newcommand{\ccbysadetitle}{%
	Lizenz: \href{http://creativecommons.org/licenses/by-sa/3.0/deed.de}
	{CC-BY-SA 3.0 \ccbysa}
}
`` in your main
% document and start with ``\title`` and ``\begin{document}`` afterwards.

% If you need to add additional packages, I recommend not doing this in this
% file, but in your main document. That way, you can just drop in a new
% ``header.tex`` and get all the new commands without having to merge manually.

% Since this file encorporates a CC-BY-SA fragment, this whole files is
% licensed under the CC-BY-SA license.

\documentclass[11pt, ngerman, fleqn, DIV=15]{scrartcl}

\usepackage{graphicx}

\setkomafont{caption}{\sffamily}
\setkomafont{captionlabel}{\usekomafont{caption}}

%%%%%%%%%%%%%%%%%%%%%%%%%%%%%%%%%%%%%%%%%%%%%%%%%%%%%%%%%%%%%%%%%%%%%%%%%%%%%%%
%                                Locale, date                                 %
%%%%%%%%%%%%%%%%%%%%%%%%%%%%%%%%%%%%%%%%%%%%%%%%%%%%%%%%%%%%%%%%%%%%%%%%%%%%%%%

\usepackage{babel}
\usepackage[iso]{isodate}

%%%%%%%%%%%%%%%%%%%%%%%%%%%%%%%%%%%%%%%%%%%%%%%%%%%%%%%%%%%%%%%%%%%%%%%%%%%%%%%
%                          Margins and other spacing                          %
%%%%%%%%%%%%%%%%%%%%%%%%%%%%%%%%%%%%%%%%%%%%%%%%%%%%%%%%%%%%%%%%%%%%%%%%%%%%%%%

\usepackage[parfill]{parskip}
\usepackage{setspace}
\usepackage[activate]{microtype}

\setlength{\columnsep}{2cm}

%%%%%%%%%%%%%%%%%%%%%%%%%%%%%%%%%%%%%%%%%%%%%%%%%%%%%%%%%%%%%%%%%%%%%%%%%%%%%%%
%                                    Color                                    %
%%%%%%%%%%%%%%%%%%%%%%%%%%%%%%%%%%%%%%%%%%%%%%%%%%%%%%%%%%%%%%%%%%%%%%%%%%%%%%%

\usepackage[usenames, dvipsnames]{xcolor}

\colorlet{darkred}{red!70!black}
\colorlet{darkblue}{blue!70!black}
\colorlet{darkgreen}{green!40!black}

%%%%%%%%%%%%%%%%%%%%%%%%%%%%%%%%%%%%%%%%%%%%%%%%%%%%%%%%%%%%%%%%%%%%%%%%%%%%%%%
%                         Font and font like settings                         %
%%%%%%%%%%%%%%%%%%%%%%%%%%%%%%%%%%%%%%%%%%%%%%%%%%%%%%%%%%%%%%%%%%%%%%%%%%%%%%%

% This replaces all fonts with Bitstream Charter, Bitstream Vera Sans and
% Bitstream Vera Mono. Math will be rendered in Charter.
\usepackage[charter, greekuppercase=italicized]{mathdesign}
\usepackage{beramono}
\usepackage{berasans}

% Bold, sans-serif tensors. This fragment is taken from “egreg” from
% http://tex.stackexchange.com/a/82747/8945 and licensed under `CC-BY-SA
% <https://creativecommons.org/licenses/by-sa/3.0/>`_.
\usepackage{bm}
\DeclareMathAlphabet{\mathsfit}{\encodingdefault}{\sfdefault}{m}{sl}
\SetMathAlphabet{\mathsfit}{bold}{\encodingdefault}{\sfdefault}{bx}{sl}
\newcommand{\tens}[1]{\bm{\mathsfit{#1}}}

% Bold vectors.
\renewcommand{\vec}[1]{\boldsymbol{#1}}

%%%%%%%%%%%%%%%%%%%%%%%%%%%%%%%%%%%%%%%%%%%%%%%%%%%%%%%%%%%%%%%%%%%%%%%%%%%%%%%
%                               Input encoding                                %
%%%%%%%%%%%%%%%%%%%%%%%%%%%%%%%%%%%%%%%%%%%%%%%%%%%%%%%%%%%%%%%%%%%%%%%%%%%%%%%

\usepackage[T1]{fontenc}
\usepackage[utf8]{inputenc}

%%%%%%%%%%%%%%%%%%%%%%%%%%%%%%%%%%%%%%%%%%%%%%%%%%%%%%%%%%%%%%%%%%%%%%%%%%%%%%%
%                         Hyperrefs and PDF metadata                          %
%%%%%%%%%%%%%%%%%%%%%%%%%%%%%%%%%%%%%%%%%%%%%%%%%%%%%%%%%%%%%%%%%%%%%%%%%%%%%%%

\usepackage{hyperref}

% This sets the author in the properties of the PDF as well. If you want to
% change it, just override it with another ``\hypersetup`` call.
\hypersetup{
	breaklinks=false,
	citecolor=darkgreen,
	colorlinks=true,
	linkcolor=darkblue,
	menucolor=black,
	pdfauthor={Martin Ueding},
	urlcolor=darkblue,
}

%%%%%%%%%%%%%%%%%%%%%%%%%%%%%%%%%%%%%%%%%%%%%%%%%%%%%%%%%%%%%%%%%%%%%%%%%%%%%%%
%                               Math Operators                                %
%%%%%%%%%%%%%%%%%%%%%%%%%%%%%%%%%%%%%%%%%%%%%%%%%%%%%%%%%%%%%%%%%%%%%%%%%%%%%%%

% AMS environments like ``align`` and theorems like ``proof``.
\usepackage{amsmath}
\usepackage{amsthm}

% Common math constructs like partial derivatives.
\usepackage{commath}

% Physical units.
\usepackage[output-decimal-marker={,}]{siunitx}

% Since I use mathdesign with italic uppercase greek characters, the Ohm unit will be displayed with an italic Ω by default. Units have to be roman, so this forces it the right way.
\DeclareSIUnit{\ohm}{$\Omegaup$}
\DeclareSIUnit{\division}{DIV}
\DeclareSIUnit{\voltss}{$\mathrm{V_{SS}}$}

% Word like operators.
\DeclareMathOperator{\acosh}{arcosh}
\DeclareMathOperator{\arcosh}{arcosh}
\DeclareMathOperator{\arcsinh}{arsinh}
\DeclareMathOperator{\arsinh}{arsinh}
\DeclareMathOperator{\asinh}{arsinh}
\DeclareMathOperator{\card}{card}
\DeclareMathOperator{\csch}{csch}
\DeclareMathOperator{\diam}{diam}
\DeclareMathOperator{\sech}{sech}
\renewcommand{\Im}{\mathop{{}\mathrm{Im}}\nolimits}
\renewcommand{\Re}{\mathop{{}\mathrm{Re}}\nolimits}

% Fourier transform.
\DeclareMathOperator{\fourier}{\ensuremath{\mathcal{F}}}

% Roman versions of “e” and “i” to serve as Euler's number and the imaginary
% constant.
\newcommand{\eup}{\mathrm e}
\newcommand{\iup}{\mathrm i}

% Symbols for the various mathematical fields (natural numbers, integers,
% rational numbers, real numbers, complex numbers).
\newcommand{\C}{\ensuremath{\mathbb C}}
\newcommand{\N}{\ensuremath{\mathbb N}}
\newcommand{\Q}{\ensuremath{\mathbb Q}}
\newcommand{\R}{\ensuremath{\mathbb R}}
\newcommand{\Z}{\ensuremath{\mathbb Z}}

% Shape like operators.
\DeclareMathOperator{\dalambert}{\Box}
\DeclareMathOperator{\laplace}{\bigtriangleup}
\newcommand{\curl}{\vnabla \times}
\newcommand{\divergence}[1]{\inner{\vnabla}{#1}}
\newcommand{\Divergence}[1]{\Inner{\vnabla}{#1}}
\newcommand{\vnabla}{\vec \nabla}

\newcommand{\half}{\frac 12}

% Unit vector (German „Einheitsvektor“).
\newcommand{\ev}{\hat{\vec e}}

% Mathematician's notation for the inner (scalar, dot) product.
\newcommand{\bracket}[1]{\langle #1 \rangle}
\newcommand{\Bracket}[1]{\left\langle #1 \right\rangle}
\newcommand{\inner}[2]{\bracket{#1, #2}}
\newcommand{\Inner}[2]{\Bracket{#1, #2}}

% Placeholders.
\newcommand{\fehlt}{\textcolor{darkred}{Hier fehlen noch Inhalte.}}
\newcommand{\messwert}{\textcolor{blue}{\square}}
\newcommand{\punkte}{\phantom{xxxxx}}

% Separator for equations on a single line.
\newcommand{\eqnsep}{,\quad}

% Quantum Mechanics.
\usepackage{braket}

% Thermodynamic partial derivative.
\newcommand\tdpd[3]{\del{\dpd{#1}{#2}}_{#3}}

%%%%%%%%%%%%%%%%%%%%%%%%%%%%%%%%%%%%%%%%%%%%%%%%%%%%%%%%%%%%%%%%%%%%%%%%%%%%%%%
%                                  Headings                                   %
%%%%%%%%%%%%%%%%%%%%%%%%%%%%%%%%%%%%%%%%%%%%%%%%%%%%%%%%%%%%%%%%%%%%%%%%%%%%%%%

% This will set fancy headings to the top of the page. The page number will be
% accompanied by the total number of pages. That way, you will know if any page
% is missing.
%
% If you do not want this for your document, you can just use
% ``\pagestyle{plain}``.

\usepackage{scrpage2}

\pagestyle{scrheadings}
\automark{section}
\chead{}
\ihead{}
\ohead{\rightmark}
\setheadsepline{.4pt}

%%%%%%%%%%%%%%%%%%%%%%%%%%%%%%%%%%%%%%%%%%%%%%%%%%%%%%%%%%%%%%%%%%%%%%%%%%%%%%%
%                            Bibliography (BibTeX)                            %
%%%%%%%%%%%%%%%%%%%%%%%%%%%%%%%%%%%%%%%%%%%%%%%%%%%%%%%%%%%%%%%%%%%%%%%%%%%%%%%

\newcommand{\bibliographyfile}{../../zentrale_BibTeX/Central}

\usepackage[
    backend=bibtex,
    style=authoryear-icomp,
    %isbn=false,
    %pagetracker=false,
    %maxbibnames=50,
    %maxcitenames=2,
    %autocite=inline,
    %block=space,
    %backref=false,
    %backrefstyle=three+,
    %date=short,
    hyperref=true
]{biblatex}

\setlength{\bibitemsep}{.7em}
\setlength{\bibhang}{4ex}

\IfFileExists{\bibliographyfile}{
    \bibliography{\bibliographyfile}
}{}

%%%%%%%%%%%%%%%%%%%%%%%%%%%%%%%%%%%%%%%%%%%%%%%%%%%%%%%%%%%%%%%%%%%%%%%%%%%%%%%
%                                Abbreviations                                %
%%%%%%%%%%%%%%%%%%%%%%%%%%%%%%%%%%%%%%%%%%%%%%%%%%%%%%%%%%%%%%%%%%%%%%%%%%%%%%%

\newcommand{\dhabk}{\mbox{d.\,h.}}

%%%%%%%%%%%%%%%%%%%%%%%%%%%%%%%%%%%%%%%%%%%%%%%%%%%%%%%%%%%%%%%%%%%%%%%%%%%%%%%
%                                  Licences                                   %
%%%%%%%%%%%%%%%%%%%%%%%%%%%%%%%%%%%%%%%%%%%%%%%%%%%%%%%%%%%%%%%%%%%%%%%%%%%%%%%

\usepackage{ccicons}

\newcommand{\ccbysadetext}{%
	\begin{small}
		Dieses Werk bzw. Inhalt steht unter einer
		\href{http://creativecommons.org/licenses/by-sa/3.0/deed.de}{%
			Creative Commons Namensnennung - Weitergabe unter gleichen
		Bedingungen 3.0 Unported Lizenz}.
	\end{small}
}

\newcommand{\ccbysadetitle}{%
	Lizenz: \href{http://creativecommons.org/licenses/by-sa/3.0/deed.de}
	{CC-BY-SA 3.0 \ccbysa}
}


%\subject{}
\title{physik521 – Übung 10}
%\subtitle{}
\author{
	Martin Ueding \\ \small{\href{mailto:mu@martin-ueding.de}{mu@martin-ueding.de}}
        \and Paul Manz
        \and Lino Lemmer \\ \small{\href{mailto:l2@uni-bonn.de}{l2@uni-bonn.de}}
}

\pagestyle{plain}

\newcommand\kB{k_\text B}
\newcommand\muB{\mu_\text B}

\begin{document}

\maketitle

%\newpage
%\tableofcontents
%\newpage

\section{Pauli-Spin-Paramagnetismus eines Elektronengases}

Die Zustandsdichte soll aus der Vorlesung bekannt sein. Wir benutzen die
„Zustandsdichte idealer Quantengase“ \parencite[]{nolting-theo6}:
\[
    \rho(E) = \begin{cases}
        (2S + 1) \frac{V}{4\piup^2} \del{\frac{2m}{\hbar^2}}^{3/2} & E \geq 0 \\
        0 & \text{sonst}.
    \end{cases}
\]

\subsection{Energie-Eigenwerte und großkanonische Zustandssumme}

Die Energie-Eigenwerte eines Elektrons mit Impuls $p$ und Spin-Quantenzahl $\sigma$ in einem Magnetfeld $B=\vec B \hat e_z$ sind
\begin{align*}
    E_{p,\sigma} &= E_{\alpha} = E_p + E_\sigma = \frac{p}{2m} + g\mu_B\sigma B.
    \intertext{%
        Hat man eine Zahl $N$ von Elektronen gegeben, berechnet sich die großkanonische Zustandssumme zu
    }
    Z_\text{G} &= \sum_{\substack{n_{\alpha_i}=0,1\\i=1,2\dots}}
    \exp\del{-\beta\del{E\del{\cbr{n_{\alpha_i}}}-\mu N}},
    \intertext{%
        mit der Gesamtenergie $E\del{\cbr{n_{\alpha_i}}} = \sum_i
        n_{\alpha_i}E_{\alpha_i}$. Dabei gilt für die Anzahl $N = \sum_i
        n_{\alpha_i}$. Damit erhält man
    }
    &= \prod_i^\infty\sum_{n_{\alpha_i}=0,1}
    \exp\del{-\beta n_{\alpha_i}\del{E_{\alpha_i}-\mu}} \\
    &= \prod_i^\infty 1 + \exp\del{-\beta\del{E_{\alpha_i}-\mu}}
\end{align*}

\subsection{Tieftemperaturentwicklung des großkanonischen Potenzials}

Das großkanonische Potenzial erhält man aus der großkanonischen Zustandssumme:
\begin{align*}
    \Omega &= -k_\text{B}T\ln Z_\text{G} \\
           &= -k_\text{B}T\sum_i^\infty\ln \del{1 + \exp\del{-\beta\del{E_{\alpha_i}-\mu}}}
    \intertext{%
        Dies kann mit Formel 5.61 aus dem Skript in ein Integral über die
        Energien und eine Summe über die Spins umgewandelt werden.
    }
    &= -k_\text{B}T\sum_\sigma\int_{-\infty}^\infty\!\dif E \,
    D_\sigma(E)\ln\del{1+\exp\del{-\beta\del{E-\mu}}}
    \intertext{%
        Nach einer partiellen Integration erhält man mit
        $\int_{-\infty}^{E}\!\dif\epsilon\,D_\sigma(E) = a_\sigma\del{E}$:
    }
    &= k_\text{B}T\sum_\sigma\int_{-\infty}^\infty\dif E \, a_\sigma\del{E}
    \frac{-\beta\exp\del{-\beta\del{E-\mu}}} {1 +
    \exp\del{-\beta\del{E-\mu}}} \\
    &= -\sum_\sigma\int_{-\infty}^\infty\!\dif E\, a_\sigma(E)f(E)
    \intertext{%
        Nach einer weiteren partiellen Integration erhält man mit
        $\int_{-\infty}^{E}\!\dif\epsilon\,a_\sigma\del{\epsilon}=b_\sigma(E)$:
    }
    &= \sum_\sigma\int_{-\infty}^\infty\!\dif E\, b_\sigma(E)f'(E)
    \intertext{%
        Eine Tieftemperaturnäherung nach Sommerfeld (Taylor-Entwicklung
        von $b$ um $E=\mu$) liefert uns
    }
    &\approx \sum_\sigma \sbr{ b\del{\mu} +
\frac{\piup^2}6D_\sigma(\mu)\del{\kB T}^2}.
\end{align*}

\subsection{Teilchenzahlen}

Die Gesamtteilchenzahl ist gegeben durch
\begin{align*}
    N &= \sum_\sigma \int_0^\infty\!\dif E \, D_\sigma(E)f(E).
    \intertext{%
        Betrachtet man die verschiedenen Spins einzeln erhält man
    }
    N_{\uparrow} &= \int_{-\infty}^\infty \!\dif E\, D_\uparrow(E)f(E).
    \intertext{%
        Mit \cite[S. 525]{nolting-theo6} ist dies
    }
    &= \half \int_{\mu_\text{B}B}^\infty\!\dif E\, D(E-\mu_\text{B}B)f(E).
    \intertext{%
        Mit $y=E-\mu_\text{B}$ und $\dif y = \dif E$ ergibt sich
    }
    &= \half \int_0^\infty\!\dif E\,
    D(y)f(y+\mu_\text{B}).
    \intertext{%
        Da $\muB B\ll1$ Taylorn wir $f$ um $y$ und erhalten
    }
    &\approx \half \int_0^\infty\!\dif E\,
    D(y)\del{f(y)+\mu_\text{B}f'(y)} \\
    &= \half \int_0^\infty\!\dif E\,
    D(y)f(y) + \frac{\mu_\text{B}B}{2} \int_0^\infty\!\dif E\,
    D(y)f'(y).
    \intertext{%
        Analog dazu die Rechnung für den anderen Spin, es folgt
    }
    N_\downarrow &= \half \int_0^\infty\!\dif E\,
    D(y)f(y) - \frac{\mu_\text{B}B}{2} \int_0^\infty\!\dif E\,
    D(y)f'(y).
\end{align*}
Dies ist auch das, was in der Abbildung zu sehen ist. Ein Spin hat mehr
Teilchen. Da aber das chemische Potenzial $\mu$ für beide gleich sein soll,
laufen einige zum anderen Spin über.

\subsection{Tieftemperaturverhalten der Magnetisierung}

Die Magnetisierung des Systems ist
\begin{align*}
    M &= \frac\muB{V}\del{N_\downarrow-N_\uparrow} \\
      &= -\frac{\muB^2B}{V}\int_0^\infty\!\dif\,D(y)f'(y).
    \intertext{%
        Eine Entwicklung nach Sommerfeld bringt für tiefe Temperaturen
    }
    &\approx \del{D(\mu) + \frac{\piup^2}6\del{\kB T}^2D''(\mu)}

\subsection{Magnetische Suszeptibilität}

\fehlt

\subsection{Feste Teilchenzahl statt festes chemisches Potenzial}

\fehlt

\section{Landau-Diamagnetismus}

\subsection{Unabhängigkeit der Zustandssumme von Magnetfeld}

Es gibt eine feste Anzahl Teilchen und verschiedene Energien, daher kann ich
den kanonischen Formalismus benutzen.

Meine Zustände sind alle Impulse $\vec p \in \R^3$. Daher muss ich für die
Zustandssumme im Impulsraum, in dem $\hat{\vec p} = \vec p$ gilt, integrieren.
\begin{align*}
    Z_{\text C, 1}
    &= \int\limits_{\R^3} \dif{^3 p} \, \exp(-\beta \hat H(\vec p)) \\
    &= \int\limits_{\R^3} \dif{^3 p} \, \exp\del{-\beta \frac1{2m} \del{\vec p + \frac ec \vec A}^2} \\
    \intertext{%
        Jetzt ist zu zeigen, dass dies unabhängig von $\vec B$ ist. Wir werden
        sogar zeigen, dass dies unabhängig von $\vec A$ ist. Dazu führen wir
        eine Substitution durch:
        \[
            \vec q := \vec p + \frac ec \vec A.
        \]
        Diese Substitution ist für alle impulsunabhängigen $\vec A$ definiert
        und es gilt vor allem $\dif{\vec q} = \dif{\vec p}$, da $\vec A$ nicht
        vom Impuls abhängt. Somit erhalten wir:
    }
    &= \int\limits_{\R^3} \dif{^3 q} \, \exp\del{-\beta \frac1{2m} \vec q^2} \\
    \intertext{%
        An dieser Stelle ist schon gezeigt, dass die Zustandssumme nicht von
        $\vec A$ abhängt. Jedoch können wir dieses Integral auch analytisch
        lösen. Wir wählen Kugelkoordinaten für den Impuls und können aufgrund
        der Isotropie den Winkelanteil schon integrieren. Es bleibt: 
    }
    &= 4 \piup \int_0^\infty \dif q \, q^2 \exp\del{-\beta \frac1{2m} q^2} \\
    \intertext{%
        Dieses Integral können wir mit partieller Integration lösen. Dabei ist
        die erste Stammfunktion zur Exponentialfunktion die Gauß'sche
        Fehlerfunktion. Die zweite Stammfunktion dazu ist beinhaltet wieder die
        Fehlerfunktion, jedoch haben wir dies nur mit Mathematica
        herausbekommen. Nach weiteren Rechnungen und weiteren Substitutionen
        erhalten wir:
    }
    &= \del{\frac{2 m \piup}\beta}^{3/2}.
\end{align*}

Wir versuchen noch, die großkanonische Zustandssumme aufzustellen:
\begin{align*}
    Z_\text{GC}
    &= \prod_p \prod_\sigma \del{1 + \exp\del{- \frac{E(p, \sigma) - \mu}{\kB
    T}}} \\
    &= \prod_p \prod_\sigma \del{1 + \exp\del{- \frac{\frac1{2m} \del{\vec p -
\frac ec \vec A}^2- \mu}{\kB
    T}}}
\end{align*}

Wenn man daraus jetzt
\[
    \exp\del{\sum_p \sum_\sigma \ln(1 + \exp(-\ldots))}
\]
formt, kann man wieder die gleiche Substitution im Integral vornehmen. Dadurch
sollte auch gezeigt sind, dass $Z_\text{GC}$ nicht von $\vec A$ abhängt.

\subsection{Entartungsgrad von Landau-Niveaus}

Der Hamiltonoperator ist gegeben als:
\[
    \hat H = \frac1{2m} \del{\hat{\vec p} + \frac ec \vec A}^2
    = \frac1{2m} \del{\hat{\vec p} + \frac ec \hat x B \ev_y}^2,
\]
wobei $\ev_y$ den Einheitsvektor in $y$-Richtung bezeichnet.

Die Zeitentwicklung von $\vec p$ ist gegeben durch:
\[
    \dod{}t \hat{\vec p} = \frac1{\iup\hbar} [\hat{\vec p}, \hat H] + \dpd{}t \hat{\vec p}.
\]

Wir setzen den Hamiltonoperator ein und benutzen die Darstellung im Impulsraum,
so dass $\hat p = p$ und $\hat x = \iup\hbar \partial/\partial p_x$ ist. Der
Impulsoperator ist nicht explizit zeitabhängig, so dass wir nur den Kommutator
ausrechnen müssen, um die Zeitentwicklung zu erhalten.

Jedoch hatte das nicht sonderlich gut geklappt.

Daher lösen wir die Schrödingergleichung mit einem Ansatz und transformieren
die verbleibende Differentialgleichung so, dass die Differentialgleichung des
harmonischen Oszillators herauskommt, wie in
\cite[Abschnitt~3.2.7]{nolting-theo6}, vorgerechnet.

Die Schrödingergleichung ist:
\begin{align*}
    E \ket\psi &= \hat H \ket\psi. \\
    \intertext{%
        Wir setzen $\hat H$ ein und multiplizieren teilweise aus, um die
        Variablen etwas zu separieren:
    }
    &= \del{\hat p_x^2 + \hat p_z^2 + \del{\hat p_y + \frac ec B \hat x}^2}
    \ket\psi. \\
    \intertext{%
        Nun wird der Ansatz $\psi(\vec r) = \exp(\iup k_y y) \exp(\iup k_z z)
        u(x)$ gewählt.
        Im Ortsraum ist der Impulsoperator eine Raumableitung. Die
        Impulsoperatoren erzeugen dadurch mit der Kettenregel jeweils ein
        $k_y^2$ und ein $k_z^2$. Für $u$ bleibt folgende Differentialgleichung
        übrig:
    }
    &= - \frac{\hbar^2}{2m} \dpd[2]{}x u(x) + \frac{\hbar^2 k_z^2}{2m} u(x) +
    \frac{1}{2m} \del{\hbar k_y + \frac ec B x}^2 u(x) \\
    \del{E - \frac{\hbar^2 k_z^2}{2m}} u(x) &= - \frac{\hbar^2}{2m} \dpd[2]{}x u(x) +
    \frac{1}{2m} \del{\hbar k_y + \frac ec B x}^2 u(x). \\
    \intertext{%
        Aus dem Ansatz „Lorentzkraft ist Zentripetalkraft“ kann die
        Zyklotronfrequenz $\omega = eB/m$ hergeleitet werden. Dies setzen wir
        ein und erhalten:
    }
    \del{E - \frac{\hbar^2 k_z^2}{2m}} u(x) &= - \frac{\hbar^2}{2m} \dpd[2]{}x u(x) +
    \frac{1}{2m} \del{\hbar k_y + \frac{m\omega}c x}^2 u(x). \\
    \intertext{%
        Der letzte Summand in der Klammer muss gerade $m \omega^2 q^2/2$
        werden, damit die Differentialgleichung die richtige Form für den
        harmonischen Oszillator bekommt. Daher fordern wir
        \[
            \frac m2 \omega^2 q^2 = \frac{1}{2m} \del{\hbar k_y +
            \frac{m\omega}c x}^2,
        \]
        was wir durch
        \[
            q = \frac 1c x + \frac{\hbar k_y}{m\omega}
        \]
        erfüllen können. Nolting hat hier keinen Faktor $1/c$. In der ganzen
        Herleitung taucht bei ihm das $c$ nicht auf. Liegt dies an natürlichen
        Einheiten mit $c = 1$? Wir müssen das $c$ ab dieser Stelle weglassen,
        da wir ansonsten nicht die richtigen Ergebnisse erhalten. Jedenfalls
        erhalten wir dann folgende Differentialgleichung für $u$:
    }
    \del{E - \frac{\hbar^2 k_z^2}{2m}} u(q) &= - \frac{\hbar^2}{2m} \dpd[2]{}q u(q) +
    \frac m2 \omega^2 q^2 u(q).
\end{align*}

Die Lösungen dieser Gleichung sind vom harmonischen Oszillator bekannt. Somit
ist gezeigt, dass die Bewegung einem harmonischen Oszillator entspricht. Mit
der Quantenzahl $n$, die durch die Hermitepolynome eingeführt wird, sind die
Energieeigenwerte durch $\hbar\omega (n + 1/2)$ gegeben. Die Energieeigenwerte
sind allerdings auch in der Klammer gegeben. Somit können wir nach $E$
auflösen und erhalten die Energieeigenwerte der Elektronen:
\[
    E = \hbar\omega \del{n + \frac12} + \frac{\hbar^2 k_z^2}{2m}.
\]

Diese Energie hängt von $n$ und $k_z$ ab, die aus $x$ und $z$ gewonnen worden
sind. $k_y$ taucht hier nicht auf, so dass die Energieeigenwerte bezüglich
$k_y$ entartet sind. Der Entartungsgrad ist die Anzahl der möglichen Werte, die
$k_y$ annehmen kann.

Da das Teilchen im Quader mit Kantenlängen $\{ L_i | i = 1, 2, 3 \}$
eingesperrt ist, darf die $x$-Koordinate nur im Intervall
\[
    - \frac{L_x}2 \leq x \leq \frac{L_x}2
\]
liegen. Mit der vorhin benutzen Substitution können wir dies durch $q$ ersetzen
und erhalten:
\[
    - \frac{L_x}2 \leq q - \frac{\hbar k_y}{eB} \leq \frac{L_x}2.
\]

Wir subtrahieren auf allen drei „Seiten“ $q$ und multiplizieren mit $-1$. Da
dies eine streng monoton fallende Transformationsfunktion ist, kehren sich die
$\leq$ zu $\geq$ um. Wir drehen dann allerdings die Gleichung wieder um, so
dass wir die ursprüngliche Form erhalten. Somit wird die Gleichung zu:
\[
    q - \frac{L_x}2 \leq \frac{\hbar k_y}{eB} \leq q + \frac{L_x}2.
\]

Dies bestimmt das maximale und minimale $k_y$. Wir haben also ein Spektrum von
\[
    \Deltaup k_y = \frac1\hbar e B L_x
\]
im Phasenraum zur Verfügung. Die Zustände sind im Phasenraum mit der
Quantenzahl $m_y$ quantisiert:
\[
    k = \frac{2\piup}{L_y} m_y.
\]

Aufgrund des linearen Zusammenhangs können wir mit $\Deltaup k$ die Anzahl der
Zustände, $\Deltaup m_y$, errechnen und erhalten so:
\[
    \Deltaup m_y = \frac{e B L_x L_y}{2 \piup \hbar}.
\]

\subsection{Zustandsdichte des Elektronengases im Magnetfeld}

\fehlt

\subsection{Großkanonisches Potenzial}

Das großkanonische Potential können wir aus der Zustandssumme der ersten
Teilaufgabe errechnen, wobei wir die Summe über die beiden entarteten Spins
direkt ausführen:
\[
    \Omega = - 2 \kB T \sum_\alpha \ln\del{1 + \frac{E_\alpha - \mu}{\kB T}}.
\]

In $\sum_\alpha$ werden auch entartete Zustände gemäß ihrem Entartungsgrad
mehrfach gezählt. Diese Summe kann man generell durch ein Integral über die
Zustandsdichte schreiben, die so (mit $\delta$-Distributionen) konstruiert ist,
dass sie bei jedem Energieeigenwert nach Integration den entsprechenden
Entartungsgrad ergibt. Somit können wir schreiben:
\[
    \Omega = - 2 \kB T \int_{-\infty}^\infty \dif E \, \rho(E) \ln\del{1 + \frac{E_\alpha - \mu}{\kB T}}.
\]

Wir gehen davon aus, dass die Zustandsdichte, die in der Teilaufgabe (c)
gegeben ist, genau die ist, die hier funktioniert. Nun können wir die
Sommerfeldentwicklung aus dem Skript benutzen, da wir hier nicht noch einmal
alle Teilschritte – zweifache partielle Integration, Reihenentwickung und
Fermiintegrale – durchrechnen möchten. Als Endergebnis ist im Skript als Formel
(5.71) angegeben:
\[
    \Omega = - 2 b(\mu) - \frac{\piup^2}3 \rho(\mu) (\kB T)^2 + \mathcal
    O(T^4).
\]

$b$ ist dabei die zweifache Stammfunktion von $\rho$, die als Parameter
stehengelassen werden durfte.

\subsection{Magnetisierung des Elektronengases}

\fehlt

\section{Bose-Einstein-Kondensation}

\subsection{Entartungsgrad und Zustandsdichte}

\fehlt

\subsection{Thermodynamischer Limes}

\fehlt

\subsection{BEC-Temperatur und Besetzungszahl des Grundzustandes}

\fehlt

\subsection{Mittlere quadratische Ausdehnung}

\fehlt

\subsection{Mittlere quadratische Ausdehnung bei Folgen der Boltzmann-Statistik}

\fehlt

\IfFileExists{\bibliographyfile}{
    \printbibliography
}{}

\end{document}

% vim: spell spelllang=de tw=79
