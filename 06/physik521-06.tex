% Copyright © 2012-2014 Martin Ueding <dev@martin-ueding.de>

% This is my general purpose LaTeX header file for writing German documents.
% Ideally, you include this using a simple ``% Copyright © 2012-2014 Martin Ueding <dev@martin-ueding.de>

% This is my general purpose LaTeX header file for writing German documents.
% Ideally, you include this using a simple ``% Copyright © 2012-2014 Martin Ueding <dev@martin-ueding.de>

% This is my general purpose LaTeX header file for writing German documents.
% Ideally, you include this using a simple ``\input{header.tex}`` in your main
% document and start with ``\title`` and ``\begin{document}`` afterwards.

% If you need to add additional packages, I recommend not doing this in this
% file, but in your main document. That way, you can just drop in a new
% ``header.tex`` and get all the new commands without having to merge manually.

% Since this file encorporates a CC-BY-SA fragment, this whole files is
% licensed under the CC-BY-SA license.

\documentclass[11pt, ngerman, fleqn, DIV=15]{scrartcl}

\usepackage{graphicx}

\setkomafont{caption}{\sffamily}
\setkomafont{captionlabel}{\usekomafont{caption}}

%%%%%%%%%%%%%%%%%%%%%%%%%%%%%%%%%%%%%%%%%%%%%%%%%%%%%%%%%%%%%%%%%%%%%%%%%%%%%%%
%                                Locale, date                                 %
%%%%%%%%%%%%%%%%%%%%%%%%%%%%%%%%%%%%%%%%%%%%%%%%%%%%%%%%%%%%%%%%%%%%%%%%%%%%%%%

\usepackage{babel}
\usepackage[iso]{isodate}

%%%%%%%%%%%%%%%%%%%%%%%%%%%%%%%%%%%%%%%%%%%%%%%%%%%%%%%%%%%%%%%%%%%%%%%%%%%%%%%
%                          Margins and other spacing                          %
%%%%%%%%%%%%%%%%%%%%%%%%%%%%%%%%%%%%%%%%%%%%%%%%%%%%%%%%%%%%%%%%%%%%%%%%%%%%%%%

\usepackage[parfill]{parskip}
\usepackage{setspace}
\usepackage[activate]{microtype}

\setlength{\columnsep}{2cm}

%%%%%%%%%%%%%%%%%%%%%%%%%%%%%%%%%%%%%%%%%%%%%%%%%%%%%%%%%%%%%%%%%%%%%%%%%%%%%%%
%                                    Color                                    %
%%%%%%%%%%%%%%%%%%%%%%%%%%%%%%%%%%%%%%%%%%%%%%%%%%%%%%%%%%%%%%%%%%%%%%%%%%%%%%%

\usepackage[usenames, dvipsnames]{xcolor}

\colorlet{darkred}{red!70!black}
\colorlet{darkblue}{blue!70!black}
\colorlet{darkgreen}{green!40!black}

%%%%%%%%%%%%%%%%%%%%%%%%%%%%%%%%%%%%%%%%%%%%%%%%%%%%%%%%%%%%%%%%%%%%%%%%%%%%%%%
%                         Font and font like settings                         %
%%%%%%%%%%%%%%%%%%%%%%%%%%%%%%%%%%%%%%%%%%%%%%%%%%%%%%%%%%%%%%%%%%%%%%%%%%%%%%%

% This replaces all fonts with Bitstream Charter, Bitstream Vera Sans and
% Bitstream Vera Mono. Math will be rendered in Charter.
\usepackage[charter, greekuppercase=italicized]{mathdesign}
\usepackage{beramono}
\usepackage{berasans}

% Bold, sans-serif tensors. This fragment is taken from “egreg” from
% http://tex.stackexchange.com/a/82747/8945 and licensed under `CC-BY-SA
% <https://creativecommons.org/licenses/by-sa/3.0/>`_.
\usepackage{bm}
\DeclareMathAlphabet{\mathsfit}{\encodingdefault}{\sfdefault}{m}{sl}
\SetMathAlphabet{\mathsfit}{bold}{\encodingdefault}{\sfdefault}{bx}{sl}
\newcommand{\tens}[1]{\bm{\mathsfit{#1}}}

% Bold vectors.
\renewcommand{\vec}[1]{\boldsymbol{#1}}

%%%%%%%%%%%%%%%%%%%%%%%%%%%%%%%%%%%%%%%%%%%%%%%%%%%%%%%%%%%%%%%%%%%%%%%%%%%%%%%
%                               Input encoding                                %
%%%%%%%%%%%%%%%%%%%%%%%%%%%%%%%%%%%%%%%%%%%%%%%%%%%%%%%%%%%%%%%%%%%%%%%%%%%%%%%

\usepackage[T1]{fontenc}
\usepackage[utf8]{inputenc}

%%%%%%%%%%%%%%%%%%%%%%%%%%%%%%%%%%%%%%%%%%%%%%%%%%%%%%%%%%%%%%%%%%%%%%%%%%%%%%%
%                         Hyperrefs and PDF metadata                          %
%%%%%%%%%%%%%%%%%%%%%%%%%%%%%%%%%%%%%%%%%%%%%%%%%%%%%%%%%%%%%%%%%%%%%%%%%%%%%%%

\usepackage{hyperref}

% This sets the author in the properties of the PDF as well. If you want to
% change it, just override it with another ``\hypersetup`` call.
\hypersetup{
	breaklinks=false,
	citecolor=darkgreen,
	colorlinks=true,
	linkcolor=darkblue,
	menucolor=black,
	pdfauthor={Martin Ueding},
	urlcolor=darkblue,
}

%%%%%%%%%%%%%%%%%%%%%%%%%%%%%%%%%%%%%%%%%%%%%%%%%%%%%%%%%%%%%%%%%%%%%%%%%%%%%%%
%                               Math Operators                                %
%%%%%%%%%%%%%%%%%%%%%%%%%%%%%%%%%%%%%%%%%%%%%%%%%%%%%%%%%%%%%%%%%%%%%%%%%%%%%%%

% AMS environments like ``align`` and theorems like ``proof``.
\usepackage{amsmath}
\usepackage{amsthm}

% Common math constructs like partial derivatives.
\usepackage{commath}

% Physical units.
\usepackage[output-decimal-marker={,}]{siunitx}

% Since I use mathdesign with italic uppercase greek characters, the Ohm unit will be displayed with an italic Ω by default. Units have to be roman, so this forces it the right way.
\DeclareSIUnit{\ohm}{$\Omegaup$}
\DeclareSIUnit{\division}{DIV}
\DeclareSIUnit{\voltss}{$\mathrm{V_{SS}}$}

% Word like operators.
\DeclareMathOperator{\acosh}{arcosh}
\DeclareMathOperator{\arcosh}{arcosh}
\DeclareMathOperator{\arcsinh}{arsinh}
\DeclareMathOperator{\arsinh}{arsinh}
\DeclareMathOperator{\asinh}{arsinh}
\DeclareMathOperator{\card}{card}
\DeclareMathOperator{\csch}{csch}
\DeclareMathOperator{\diam}{diam}
\DeclareMathOperator{\sech}{sech}
\renewcommand{\Im}{\mathop{{}\mathrm{Im}}\nolimits}
\renewcommand{\Re}{\mathop{{}\mathrm{Re}}\nolimits}

% Fourier transform.
\DeclareMathOperator{\fourier}{\ensuremath{\mathcal{F}}}

% Roman versions of “e” and “i” to serve as Euler's number and the imaginary
% constant.
\newcommand{\eup}{\mathrm e}
\newcommand{\iup}{\mathrm i}

% Symbols for the various mathematical fields (natural numbers, integers,
% rational numbers, real numbers, complex numbers).
\newcommand{\C}{\ensuremath{\mathbb C}}
\newcommand{\N}{\ensuremath{\mathbb N}}
\newcommand{\Q}{\ensuremath{\mathbb Q}}
\newcommand{\R}{\ensuremath{\mathbb R}}
\newcommand{\Z}{\ensuremath{\mathbb Z}}

% Shape like operators.
\DeclareMathOperator{\dalambert}{\Box}
\DeclareMathOperator{\laplace}{\bigtriangleup}
\newcommand{\curl}{\vnabla \times}
\newcommand{\divergence}[1]{\inner{\vnabla}{#1}}
\newcommand{\Divergence}[1]{\Inner{\vnabla}{#1}}
\newcommand{\vnabla}{\vec \nabla}

\newcommand{\half}{\frac 12}

% Unit vector (German „Einheitsvektor“).
\newcommand{\ev}{\hat{\vec e}}

% Mathematician's notation for the inner (scalar, dot) product.
\newcommand{\bracket}[1]{\langle #1 \rangle}
\newcommand{\Bracket}[1]{\left\langle #1 \right\rangle}
\newcommand{\inner}[2]{\bracket{#1, #2}}
\newcommand{\Inner}[2]{\Bracket{#1, #2}}

% Placeholders.
\newcommand{\fehlt}{\textcolor{darkred}{Hier fehlen noch Inhalte.}}
\newcommand{\messwert}{\textcolor{blue}{\square}}
\newcommand{\punkte}{\phantom{xxxxx}}

% Separator for equations on a single line.
\newcommand{\eqnsep}{,\quad}

% Quantum Mechanics.
\usepackage{braket}

% Thermodynamic partial derivative.
\newcommand\tdpd[3]{\del{\dpd{#1}{#2}}_{#3}}

%%%%%%%%%%%%%%%%%%%%%%%%%%%%%%%%%%%%%%%%%%%%%%%%%%%%%%%%%%%%%%%%%%%%%%%%%%%%%%%
%                                  Headings                                   %
%%%%%%%%%%%%%%%%%%%%%%%%%%%%%%%%%%%%%%%%%%%%%%%%%%%%%%%%%%%%%%%%%%%%%%%%%%%%%%%

% This will set fancy headings to the top of the page. The page number will be
% accompanied by the total number of pages. That way, you will know if any page
% is missing.
%
% If you do not want this for your document, you can just use
% ``\pagestyle{plain}``.

\usepackage{scrpage2}

\pagestyle{scrheadings}
\automark{section}
\chead{}
\ihead{}
\ohead{\rightmark}
\setheadsepline{.4pt}

%%%%%%%%%%%%%%%%%%%%%%%%%%%%%%%%%%%%%%%%%%%%%%%%%%%%%%%%%%%%%%%%%%%%%%%%%%%%%%%
%                            Bibliography (BibTeX)                            %
%%%%%%%%%%%%%%%%%%%%%%%%%%%%%%%%%%%%%%%%%%%%%%%%%%%%%%%%%%%%%%%%%%%%%%%%%%%%%%%

\newcommand{\bibliographyfile}{../../zentrale_BibTeX/Central}

\usepackage[
    backend=bibtex,
    style=authoryear-icomp,
    %isbn=false,
    %pagetracker=false,
    %maxbibnames=50,
    %maxcitenames=2,
    %autocite=inline,
    %block=space,
    %backref=false,
    %backrefstyle=three+,
    %date=short,
    hyperref=true
]{biblatex}

\setlength{\bibitemsep}{.7em}
\setlength{\bibhang}{4ex}

\IfFileExists{\bibliographyfile}{
    \bibliography{\bibliographyfile}
}{}

%%%%%%%%%%%%%%%%%%%%%%%%%%%%%%%%%%%%%%%%%%%%%%%%%%%%%%%%%%%%%%%%%%%%%%%%%%%%%%%
%                                Abbreviations                                %
%%%%%%%%%%%%%%%%%%%%%%%%%%%%%%%%%%%%%%%%%%%%%%%%%%%%%%%%%%%%%%%%%%%%%%%%%%%%%%%

\newcommand{\dhabk}{\mbox{d.\,h.}}

%%%%%%%%%%%%%%%%%%%%%%%%%%%%%%%%%%%%%%%%%%%%%%%%%%%%%%%%%%%%%%%%%%%%%%%%%%%%%%%
%                                  Licences                                   %
%%%%%%%%%%%%%%%%%%%%%%%%%%%%%%%%%%%%%%%%%%%%%%%%%%%%%%%%%%%%%%%%%%%%%%%%%%%%%%%

\usepackage{ccicons}

\newcommand{\ccbysadetext}{%
	\begin{small}
		Dieses Werk bzw. Inhalt steht unter einer
		\href{http://creativecommons.org/licenses/by-sa/3.0/deed.de}{%
			Creative Commons Namensnennung - Weitergabe unter gleichen
		Bedingungen 3.0 Unported Lizenz}.
	\end{small}
}

\newcommand{\ccbysadetitle}{%
	Lizenz: \href{http://creativecommons.org/licenses/by-sa/3.0/deed.de}
	{CC-BY-SA 3.0 \ccbysa}
}
`` in your main
% document and start with ``\title`` and ``\begin{document}`` afterwards.

% If you need to add additional packages, I recommend not doing this in this
% file, but in your main document. That way, you can just drop in a new
% ``header.tex`` and get all the new commands without having to merge manually.

% Since this file encorporates a CC-BY-SA fragment, this whole files is
% licensed under the CC-BY-SA license.

\documentclass[11pt, ngerman, fleqn, DIV=15]{scrartcl}

\usepackage{graphicx}

\setkomafont{caption}{\sffamily}
\setkomafont{captionlabel}{\usekomafont{caption}}

%%%%%%%%%%%%%%%%%%%%%%%%%%%%%%%%%%%%%%%%%%%%%%%%%%%%%%%%%%%%%%%%%%%%%%%%%%%%%%%
%                                Locale, date                                 %
%%%%%%%%%%%%%%%%%%%%%%%%%%%%%%%%%%%%%%%%%%%%%%%%%%%%%%%%%%%%%%%%%%%%%%%%%%%%%%%

\usepackage{babel}
\usepackage[iso]{isodate}

%%%%%%%%%%%%%%%%%%%%%%%%%%%%%%%%%%%%%%%%%%%%%%%%%%%%%%%%%%%%%%%%%%%%%%%%%%%%%%%
%                          Margins and other spacing                          %
%%%%%%%%%%%%%%%%%%%%%%%%%%%%%%%%%%%%%%%%%%%%%%%%%%%%%%%%%%%%%%%%%%%%%%%%%%%%%%%

\usepackage[parfill]{parskip}
\usepackage{setspace}
\usepackage[activate]{microtype}

\setlength{\columnsep}{2cm}

%%%%%%%%%%%%%%%%%%%%%%%%%%%%%%%%%%%%%%%%%%%%%%%%%%%%%%%%%%%%%%%%%%%%%%%%%%%%%%%
%                                    Color                                    %
%%%%%%%%%%%%%%%%%%%%%%%%%%%%%%%%%%%%%%%%%%%%%%%%%%%%%%%%%%%%%%%%%%%%%%%%%%%%%%%

\usepackage[usenames, dvipsnames]{xcolor}

\colorlet{darkred}{red!70!black}
\colorlet{darkblue}{blue!70!black}
\colorlet{darkgreen}{green!40!black}

%%%%%%%%%%%%%%%%%%%%%%%%%%%%%%%%%%%%%%%%%%%%%%%%%%%%%%%%%%%%%%%%%%%%%%%%%%%%%%%
%                         Font and font like settings                         %
%%%%%%%%%%%%%%%%%%%%%%%%%%%%%%%%%%%%%%%%%%%%%%%%%%%%%%%%%%%%%%%%%%%%%%%%%%%%%%%

% This replaces all fonts with Bitstream Charter, Bitstream Vera Sans and
% Bitstream Vera Mono. Math will be rendered in Charter.
\usepackage[charter, greekuppercase=italicized]{mathdesign}
\usepackage{beramono}
\usepackage{berasans}

% Bold, sans-serif tensors. This fragment is taken from “egreg” from
% http://tex.stackexchange.com/a/82747/8945 and licensed under `CC-BY-SA
% <https://creativecommons.org/licenses/by-sa/3.0/>`_.
\usepackage{bm}
\DeclareMathAlphabet{\mathsfit}{\encodingdefault}{\sfdefault}{m}{sl}
\SetMathAlphabet{\mathsfit}{bold}{\encodingdefault}{\sfdefault}{bx}{sl}
\newcommand{\tens}[1]{\bm{\mathsfit{#1}}}

% Bold vectors.
\renewcommand{\vec}[1]{\boldsymbol{#1}}

%%%%%%%%%%%%%%%%%%%%%%%%%%%%%%%%%%%%%%%%%%%%%%%%%%%%%%%%%%%%%%%%%%%%%%%%%%%%%%%
%                               Input encoding                                %
%%%%%%%%%%%%%%%%%%%%%%%%%%%%%%%%%%%%%%%%%%%%%%%%%%%%%%%%%%%%%%%%%%%%%%%%%%%%%%%

\usepackage[T1]{fontenc}
\usepackage[utf8]{inputenc}

%%%%%%%%%%%%%%%%%%%%%%%%%%%%%%%%%%%%%%%%%%%%%%%%%%%%%%%%%%%%%%%%%%%%%%%%%%%%%%%
%                         Hyperrefs and PDF metadata                          %
%%%%%%%%%%%%%%%%%%%%%%%%%%%%%%%%%%%%%%%%%%%%%%%%%%%%%%%%%%%%%%%%%%%%%%%%%%%%%%%

\usepackage{hyperref}

% This sets the author in the properties of the PDF as well. If you want to
% change it, just override it with another ``\hypersetup`` call.
\hypersetup{
	breaklinks=false,
	citecolor=darkgreen,
	colorlinks=true,
	linkcolor=darkblue,
	menucolor=black,
	pdfauthor={Martin Ueding},
	urlcolor=darkblue,
}

%%%%%%%%%%%%%%%%%%%%%%%%%%%%%%%%%%%%%%%%%%%%%%%%%%%%%%%%%%%%%%%%%%%%%%%%%%%%%%%
%                               Math Operators                                %
%%%%%%%%%%%%%%%%%%%%%%%%%%%%%%%%%%%%%%%%%%%%%%%%%%%%%%%%%%%%%%%%%%%%%%%%%%%%%%%

% AMS environments like ``align`` and theorems like ``proof``.
\usepackage{amsmath}
\usepackage{amsthm}

% Common math constructs like partial derivatives.
\usepackage{commath}

% Physical units.
\usepackage[output-decimal-marker={,}]{siunitx}

% Since I use mathdesign with italic uppercase greek characters, the Ohm unit will be displayed with an italic Ω by default. Units have to be roman, so this forces it the right way.
\DeclareSIUnit{\ohm}{$\Omegaup$}
\DeclareSIUnit{\division}{DIV}
\DeclareSIUnit{\voltss}{$\mathrm{V_{SS}}$}

% Word like operators.
\DeclareMathOperator{\acosh}{arcosh}
\DeclareMathOperator{\arcosh}{arcosh}
\DeclareMathOperator{\arcsinh}{arsinh}
\DeclareMathOperator{\arsinh}{arsinh}
\DeclareMathOperator{\asinh}{arsinh}
\DeclareMathOperator{\card}{card}
\DeclareMathOperator{\csch}{csch}
\DeclareMathOperator{\diam}{diam}
\DeclareMathOperator{\sech}{sech}
\renewcommand{\Im}{\mathop{{}\mathrm{Im}}\nolimits}
\renewcommand{\Re}{\mathop{{}\mathrm{Re}}\nolimits}

% Fourier transform.
\DeclareMathOperator{\fourier}{\ensuremath{\mathcal{F}}}

% Roman versions of “e” and “i” to serve as Euler's number and the imaginary
% constant.
\newcommand{\eup}{\mathrm e}
\newcommand{\iup}{\mathrm i}

% Symbols for the various mathematical fields (natural numbers, integers,
% rational numbers, real numbers, complex numbers).
\newcommand{\C}{\ensuremath{\mathbb C}}
\newcommand{\N}{\ensuremath{\mathbb N}}
\newcommand{\Q}{\ensuremath{\mathbb Q}}
\newcommand{\R}{\ensuremath{\mathbb R}}
\newcommand{\Z}{\ensuremath{\mathbb Z}}

% Shape like operators.
\DeclareMathOperator{\dalambert}{\Box}
\DeclareMathOperator{\laplace}{\bigtriangleup}
\newcommand{\curl}{\vnabla \times}
\newcommand{\divergence}[1]{\inner{\vnabla}{#1}}
\newcommand{\Divergence}[1]{\Inner{\vnabla}{#1}}
\newcommand{\vnabla}{\vec \nabla}

\newcommand{\half}{\frac 12}

% Unit vector (German „Einheitsvektor“).
\newcommand{\ev}{\hat{\vec e}}

% Mathematician's notation for the inner (scalar, dot) product.
\newcommand{\bracket}[1]{\langle #1 \rangle}
\newcommand{\Bracket}[1]{\left\langle #1 \right\rangle}
\newcommand{\inner}[2]{\bracket{#1, #2}}
\newcommand{\Inner}[2]{\Bracket{#1, #2}}

% Placeholders.
\newcommand{\fehlt}{\textcolor{darkred}{Hier fehlen noch Inhalte.}}
\newcommand{\messwert}{\textcolor{blue}{\square}}
\newcommand{\punkte}{\phantom{xxxxx}}

% Separator for equations on a single line.
\newcommand{\eqnsep}{,\quad}

% Quantum Mechanics.
\usepackage{braket}

% Thermodynamic partial derivative.
\newcommand\tdpd[3]{\del{\dpd{#1}{#2}}_{#3}}

%%%%%%%%%%%%%%%%%%%%%%%%%%%%%%%%%%%%%%%%%%%%%%%%%%%%%%%%%%%%%%%%%%%%%%%%%%%%%%%
%                                  Headings                                   %
%%%%%%%%%%%%%%%%%%%%%%%%%%%%%%%%%%%%%%%%%%%%%%%%%%%%%%%%%%%%%%%%%%%%%%%%%%%%%%%

% This will set fancy headings to the top of the page. The page number will be
% accompanied by the total number of pages. That way, you will know if any page
% is missing.
%
% If you do not want this for your document, you can just use
% ``\pagestyle{plain}``.

\usepackage{scrpage2}

\pagestyle{scrheadings}
\automark{section}
\chead{}
\ihead{}
\ohead{\rightmark}
\setheadsepline{.4pt}

%%%%%%%%%%%%%%%%%%%%%%%%%%%%%%%%%%%%%%%%%%%%%%%%%%%%%%%%%%%%%%%%%%%%%%%%%%%%%%%
%                            Bibliography (BibTeX)                            %
%%%%%%%%%%%%%%%%%%%%%%%%%%%%%%%%%%%%%%%%%%%%%%%%%%%%%%%%%%%%%%%%%%%%%%%%%%%%%%%

\newcommand{\bibliographyfile}{../../zentrale_BibTeX/Central}

\usepackage[
    backend=bibtex,
    style=authoryear-icomp,
    %isbn=false,
    %pagetracker=false,
    %maxbibnames=50,
    %maxcitenames=2,
    %autocite=inline,
    %block=space,
    %backref=false,
    %backrefstyle=three+,
    %date=short,
    hyperref=true
]{biblatex}

\setlength{\bibitemsep}{.7em}
\setlength{\bibhang}{4ex}

\IfFileExists{\bibliographyfile}{
    \bibliography{\bibliographyfile}
}{}

%%%%%%%%%%%%%%%%%%%%%%%%%%%%%%%%%%%%%%%%%%%%%%%%%%%%%%%%%%%%%%%%%%%%%%%%%%%%%%%
%                                Abbreviations                                %
%%%%%%%%%%%%%%%%%%%%%%%%%%%%%%%%%%%%%%%%%%%%%%%%%%%%%%%%%%%%%%%%%%%%%%%%%%%%%%%

\newcommand{\dhabk}{\mbox{d.\,h.}}

%%%%%%%%%%%%%%%%%%%%%%%%%%%%%%%%%%%%%%%%%%%%%%%%%%%%%%%%%%%%%%%%%%%%%%%%%%%%%%%
%                                  Licences                                   %
%%%%%%%%%%%%%%%%%%%%%%%%%%%%%%%%%%%%%%%%%%%%%%%%%%%%%%%%%%%%%%%%%%%%%%%%%%%%%%%

\usepackage{ccicons}

\newcommand{\ccbysadetext}{%
	\begin{small}
		Dieses Werk bzw. Inhalt steht unter einer
		\href{http://creativecommons.org/licenses/by-sa/3.0/deed.de}{%
			Creative Commons Namensnennung - Weitergabe unter gleichen
		Bedingungen 3.0 Unported Lizenz}.
	\end{small}
}

\newcommand{\ccbysadetitle}{%
	Lizenz: \href{http://creativecommons.org/licenses/by-sa/3.0/deed.de}
	{CC-BY-SA 3.0 \ccbysa}
}
`` in your main
% document and start with ``\title`` and ``\begin{document}`` afterwards.

% If you need to add additional packages, I recommend not doing this in this
% file, but in your main document. That way, you can just drop in a new
% ``header.tex`` and get all the new commands without having to merge manually.

% Since this file encorporates a CC-BY-SA fragment, this whole files is
% licensed under the CC-BY-SA license.

\documentclass[11pt, ngerman, fleqn, DIV=15]{scrartcl}

\usepackage{graphicx}

\setkomafont{caption}{\sffamily}
\setkomafont{captionlabel}{\usekomafont{caption}}

%%%%%%%%%%%%%%%%%%%%%%%%%%%%%%%%%%%%%%%%%%%%%%%%%%%%%%%%%%%%%%%%%%%%%%%%%%%%%%%
%                                Locale, date                                 %
%%%%%%%%%%%%%%%%%%%%%%%%%%%%%%%%%%%%%%%%%%%%%%%%%%%%%%%%%%%%%%%%%%%%%%%%%%%%%%%

\usepackage{babel}
\usepackage[iso]{isodate}

%%%%%%%%%%%%%%%%%%%%%%%%%%%%%%%%%%%%%%%%%%%%%%%%%%%%%%%%%%%%%%%%%%%%%%%%%%%%%%%
%                          Margins and other spacing                          %
%%%%%%%%%%%%%%%%%%%%%%%%%%%%%%%%%%%%%%%%%%%%%%%%%%%%%%%%%%%%%%%%%%%%%%%%%%%%%%%

\usepackage[parfill]{parskip}
\usepackage{setspace}
\usepackage[activate]{microtype}

\setlength{\columnsep}{2cm}

%%%%%%%%%%%%%%%%%%%%%%%%%%%%%%%%%%%%%%%%%%%%%%%%%%%%%%%%%%%%%%%%%%%%%%%%%%%%%%%
%                                    Color                                    %
%%%%%%%%%%%%%%%%%%%%%%%%%%%%%%%%%%%%%%%%%%%%%%%%%%%%%%%%%%%%%%%%%%%%%%%%%%%%%%%

\usepackage[usenames, dvipsnames]{xcolor}

\colorlet{darkred}{red!70!black}
\colorlet{darkblue}{blue!70!black}
\colorlet{darkgreen}{green!40!black}

%%%%%%%%%%%%%%%%%%%%%%%%%%%%%%%%%%%%%%%%%%%%%%%%%%%%%%%%%%%%%%%%%%%%%%%%%%%%%%%
%                         Font and font like settings                         %
%%%%%%%%%%%%%%%%%%%%%%%%%%%%%%%%%%%%%%%%%%%%%%%%%%%%%%%%%%%%%%%%%%%%%%%%%%%%%%%

% This replaces all fonts with Bitstream Charter, Bitstream Vera Sans and
% Bitstream Vera Mono. Math will be rendered in Charter.
\usepackage[charter, greekuppercase=italicized]{mathdesign}
\usepackage{beramono}
\usepackage{berasans}

% Bold, sans-serif tensors. This fragment is taken from “egreg” from
% http://tex.stackexchange.com/a/82747/8945 and licensed under `CC-BY-SA
% <https://creativecommons.org/licenses/by-sa/3.0/>`_.
\usepackage{bm}
\DeclareMathAlphabet{\mathsfit}{\encodingdefault}{\sfdefault}{m}{sl}
\SetMathAlphabet{\mathsfit}{bold}{\encodingdefault}{\sfdefault}{bx}{sl}
\newcommand{\tens}[1]{\bm{\mathsfit{#1}}}

% Bold vectors.
\renewcommand{\vec}[1]{\boldsymbol{#1}}

%%%%%%%%%%%%%%%%%%%%%%%%%%%%%%%%%%%%%%%%%%%%%%%%%%%%%%%%%%%%%%%%%%%%%%%%%%%%%%%
%                               Input encoding                                %
%%%%%%%%%%%%%%%%%%%%%%%%%%%%%%%%%%%%%%%%%%%%%%%%%%%%%%%%%%%%%%%%%%%%%%%%%%%%%%%

\usepackage[T1]{fontenc}
\usepackage[utf8]{inputenc}

%%%%%%%%%%%%%%%%%%%%%%%%%%%%%%%%%%%%%%%%%%%%%%%%%%%%%%%%%%%%%%%%%%%%%%%%%%%%%%%
%                         Hyperrefs and PDF metadata                          %
%%%%%%%%%%%%%%%%%%%%%%%%%%%%%%%%%%%%%%%%%%%%%%%%%%%%%%%%%%%%%%%%%%%%%%%%%%%%%%%

\usepackage{hyperref}

% This sets the author in the properties of the PDF as well. If you want to
% change it, just override it with another ``\hypersetup`` call.
\hypersetup{
	breaklinks=false,
	citecolor=darkgreen,
	colorlinks=true,
	linkcolor=darkblue,
	menucolor=black,
	pdfauthor={Martin Ueding},
	urlcolor=darkblue,
}

%%%%%%%%%%%%%%%%%%%%%%%%%%%%%%%%%%%%%%%%%%%%%%%%%%%%%%%%%%%%%%%%%%%%%%%%%%%%%%%
%                               Math Operators                                %
%%%%%%%%%%%%%%%%%%%%%%%%%%%%%%%%%%%%%%%%%%%%%%%%%%%%%%%%%%%%%%%%%%%%%%%%%%%%%%%

% AMS environments like ``align`` and theorems like ``proof``.
\usepackage{amsmath}
\usepackage{amsthm}

% Common math constructs like partial derivatives.
\usepackage{commath}

% Physical units.
\usepackage[output-decimal-marker={,}]{siunitx}

% Since I use mathdesign with italic uppercase greek characters, the Ohm unit will be displayed with an italic Ω by default. Units have to be roman, so this forces it the right way.
\DeclareSIUnit{\ohm}{$\Omegaup$}
\DeclareSIUnit{\division}{DIV}
\DeclareSIUnit{\voltss}{$\mathrm{V_{SS}}$}

% Word like operators.
\DeclareMathOperator{\acosh}{arcosh}
\DeclareMathOperator{\arcosh}{arcosh}
\DeclareMathOperator{\arcsinh}{arsinh}
\DeclareMathOperator{\arsinh}{arsinh}
\DeclareMathOperator{\asinh}{arsinh}
\DeclareMathOperator{\card}{card}
\DeclareMathOperator{\csch}{csch}
\DeclareMathOperator{\diam}{diam}
\DeclareMathOperator{\sech}{sech}
\renewcommand{\Im}{\mathop{{}\mathrm{Im}}\nolimits}
\renewcommand{\Re}{\mathop{{}\mathrm{Re}}\nolimits}

% Fourier transform.
\DeclareMathOperator{\fourier}{\ensuremath{\mathcal{F}}}

% Roman versions of “e” and “i” to serve as Euler's number and the imaginary
% constant.
\newcommand{\eup}{\mathrm e}
\newcommand{\iup}{\mathrm i}

% Symbols for the various mathematical fields (natural numbers, integers,
% rational numbers, real numbers, complex numbers).
\newcommand{\C}{\ensuremath{\mathbb C}}
\newcommand{\N}{\ensuremath{\mathbb N}}
\newcommand{\Q}{\ensuremath{\mathbb Q}}
\newcommand{\R}{\ensuremath{\mathbb R}}
\newcommand{\Z}{\ensuremath{\mathbb Z}}

% Shape like operators.
\DeclareMathOperator{\dalambert}{\Box}
\DeclareMathOperator{\laplace}{\bigtriangleup}
\newcommand{\curl}{\vnabla \times}
\newcommand{\divergence}[1]{\inner{\vnabla}{#1}}
\newcommand{\Divergence}[1]{\Inner{\vnabla}{#1}}
\newcommand{\vnabla}{\vec \nabla}

\newcommand{\half}{\frac 12}

% Unit vector (German „Einheitsvektor“).
\newcommand{\ev}{\hat{\vec e}}

% Mathematician's notation for the inner (scalar, dot) product.
\newcommand{\bracket}[1]{\langle #1 \rangle}
\newcommand{\Bracket}[1]{\left\langle #1 \right\rangle}
\newcommand{\inner}[2]{\bracket{#1, #2}}
\newcommand{\Inner}[2]{\Bracket{#1, #2}}

% Placeholders.
\newcommand{\fehlt}{\textcolor{darkred}{Hier fehlen noch Inhalte.}}
\newcommand{\messwert}{\textcolor{blue}{\square}}
\newcommand{\punkte}{\phantom{xxxxx}}

% Separator for equations on a single line.
\newcommand{\eqnsep}{,\quad}

% Quantum Mechanics.
\usepackage{braket}

% Thermodynamic partial derivative.
\newcommand\tdpd[3]{\del{\dpd{#1}{#2}}_{#3}}

%%%%%%%%%%%%%%%%%%%%%%%%%%%%%%%%%%%%%%%%%%%%%%%%%%%%%%%%%%%%%%%%%%%%%%%%%%%%%%%
%                                  Headings                                   %
%%%%%%%%%%%%%%%%%%%%%%%%%%%%%%%%%%%%%%%%%%%%%%%%%%%%%%%%%%%%%%%%%%%%%%%%%%%%%%%

% This will set fancy headings to the top of the page. The page number will be
% accompanied by the total number of pages. That way, you will know if any page
% is missing.
%
% If you do not want this for your document, you can just use
% ``\pagestyle{plain}``.

\usepackage{scrpage2}

\pagestyle{scrheadings}
\automark{section}
\chead{}
\ihead{}
\ohead{\rightmark}
\setheadsepline{.4pt}

%%%%%%%%%%%%%%%%%%%%%%%%%%%%%%%%%%%%%%%%%%%%%%%%%%%%%%%%%%%%%%%%%%%%%%%%%%%%%%%
%                            Bibliography (BibTeX)                            %
%%%%%%%%%%%%%%%%%%%%%%%%%%%%%%%%%%%%%%%%%%%%%%%%%%%%%%%%%%%%%%%%%%%%%%%%%%%%%%%

\newcommand{\bibliographyfile}{../../zentrale_BibTeX/Central}

\usepackage[
    backend=bibtex,
    style=authoryear-icomp,
    %isbn=false,
    %pagetracker=false,
    %maxbibnames=50,
    %maxcitenames=2,
    %autocite=inline,
    %block=space,
    %backref=false,
    %backrefstyle=three+,
    %date=short,
    hyperref=true
]{biblatex}

\setlength{\bibitemsep}{.7em}
\setlength{\bibhang}{4ex}

\IfFileExists{\bibliographyfile}{
    \bibliography{\bibliographyfile}
}{}

%%%%%%%%%%%%%%%%%%%%%%%%%%%%%%%%%%%%%%%%%%%%%%%%%%%%%%%%%%%%%%%%%%%%%%%%%%%%%%%
%                                Abbreviations                                %
%%%%%%%%%%%%%%%%%%%%%%%%%%%%%%%%%%%%%%%%%%%%%%%%%%%%%%%%%%%%%%%%%%%%%%%%%%%%%%%

\newcommand{\dhabk}{\mbox{d.\,h.}}

%%%%%%%%%%%%%%%%%%%%%%%%%%%%%%%%%%%%%%%%%%%%%%%%%%%%%%%%%%%%%%%%%%%%%%%%%%%%%%%
%                                  Licences                                   %
%%%%%%%%%%%%%%%%%%%%%%%%%%%%%%%%%%%%%%%%%%%%%%%%%%%%%%%%%%%%%%%%%%%%%%%%%%%%%%%

\usepackage{ccicons}

\newcommand{\ccbysadetext}{%
	\begin{small}
		Dieses Werk bzw. Inhalt steht unter einer
		\href{http://creativecommons.org/licenses/by-sa/3.0/deed.de}{%
			Creative Commons Namensnennung - Weitergabe unter gleichen
		Bedingungen 3.0 Unported Lizenz}.
	\end{small}
}

\newcommand{\ccbysadetitle}{%
	Lizenz: \href{http://creativecommons.org/licenses/by-sa/3.0/deed.de}
	{CC-BY-SA 3.0 \ccbysa}
}


\setcounter{section}{0}
\renewcommand\thesection{H\,6.\arabic{section}}
\renewcommand\thesubsection{\thesection.\alph{subsection}}

\title{physik521: Übungsblatt 06}
\author{%
    Lino Lemmer \\ \small{\texttt{s6lilemm@uni-bonn.de}}
    \and
    Martin Ueding \\ \small{\texttt{mu@martin-ueding.de}}
    \and
    Paul Manz \\ \small{\texttt{p.m@uni-bonn.de}}
}

\begin{document}
\maketitle

\section{Energiefluktuationen im kanonischen Ensemble}

Zunächst gilt generell Folgendes:
\[
    \mw{(\Delta E)^2} = \mw{E^2} - \mw{E}^2.
\]

Wir bilden die Ableitung von $\mw E$ und fassen Mittelwertsbildungen zusammen:
\begin{align*}
    kT^2 \dpd{\mw E}T
    &= \frac{\sum_n \exp\del{-\frac{E_n}{kT}} E_n^2 \cdot Z_K - Z_K \cdot \mw E \sum_j \exp\del{-\frac{E_j}{kT}} E_j}{Z_K^2} \\
    &= \mw{E^2} - \mw{E}^2 \\
    &= \mw{(\Delta E)^2}
\end{align*}

Somit ist die Relation gezeigt.

Dies hängt mit der Wärmekapazität zusammen, da diese wie folgt definiert ist:
\[
    c_V = \tdpd ETV.
\]
Es unterscheidet sich also nur um einen Faktor $kT^2$.

Laut Skript sollen im Grenzfall $N \to \infty$ die Fluktuationen gegen 0 gehen.
Daraus soll dann folgen, dass der mikrokanonische und kanonische Formalismus
equivalent sind.

\end{document}
